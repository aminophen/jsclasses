% ^^A -*- japanese-latex -*-
%
\ifx\epTeXinputencoding\undefined\else
  \epTeXinputencoding utf8 % ^^A added (2017-10-04)
\fi
%
% \iffalse
%
%   okumacro.dtx
%   私家版マクロ集
%   okumura@okumuralab.org
%   http://oku.edu.mie-u.ac.jp/~okumura/
%
%<okumacro>\NeedsTeXFormat{pLaTeX2e}
%<okumacro>\ProvidesPackage{okumacro}[2025/05/10 okumura, texjporg]
%<*driver>
\RequirePackage{plautopatch}
\documentclass[dvipdfmx]{jsarticle}
\usepackage{doc}
\xspcode"5C=1 %% \
\xspcode"22=1 %% "
\usepackage{okumacro}
\addtolength{\textwidth}{-1in}
\addtolength{\evensidemargin}{1in}
\addtolength{\oddsidemargin}{1in}
\addtolength{\marginparwidth}{1in}
\setlength\marginparsep{5pt}
\setlength\marginparpush{0pt}
% \OnlyDescription
\CodelineNumbered
\DisableCrossrefs
\setcounter{StandardModuleDepth}{1}
\GetFileInfo{okumacro.sty}
\begin{document}
  \DocInput{okumacro.dtx}
\end{document}
%</driver>
%
% \fi
%
% \title{私家版マクロ集}
% \author{奥村晴彦}
% \date{\filedate}
% \maketitle
%
% \MakeShortVerb{\|}
%
% \section{はじめに}
%
% これは『\pLaTeXe 美文書作成入門』などの本を作ったときに使ったマクロを
% 集めたものです。作っただけでテストもしていないものなども混じっている
% 可能性があります。概して他の方にはあまり利用価値がないものです。
%
% [2002-12-19] いろいろなものに収録していただく際にライセンスを明確にする
% 必要が生じてきました。アスキーのものが最近は(modified)BSDライセンスになっ
% ていますので,私のものもそれに準じて(modified)BSDとすることにします。
%
% [2016-07-30] 従来は,ascmac→okumacroの順に読み込んだ場合にエラーが出て
% いましたが,新しい仕様では,ascmacとokumacroの両方を読み込む場合には
% 「後に読み込んだほう」が勝ちます。
% 具体的には |\keytop|,|\return|,screen環境,shadebox環境があります。
%
% \section{数値集}
%
% \begin{itemize}
% \item B5: $182\,\mathrm{mm} \times 257\,\mathrm{mm}$
% \item B4: $257\,\mathrm{mm} \times 364\,\mathrm{mm}$
% \item A5: $148\,\mathrm{mm} \times 210\,\mathrm{mm}$
% \item A4: $210\,\mathrm{mm} \times 297\,\mathrm{mm}$
% \item $1\,\mathrm{in} = 72.27\,\mathrm{pt} = 2.54\,\mathrm{cm}$
% \item $1\,\mathrm{Q} = 0.25\,\mathrm{mm} = 0.711319\,\mathrm{pt}$
% \item $1\,\mathrm{pt} = 1.405839\,\mathrm{Q} = 0.3514598\,\mathrm{mm}$
% \item min10: $1\,\mathrm{zw} = 9.62216\,\mathrm{pt}
%       = 3.3818\,\mathrm{mm} = 13.5272\,\mathrm{Q}$,
%       $1\,\mathrm{zh} = 9.16443\,\mathrm{pt}$
% \item cmr10: $1\,\mathrm{em} = 10\,\mathrm{pt}$,
%       $1\,\mathrm{ex} = 4.30554\,\mathrm{pt}$
% \end{itemize}
%
% \StopEventually{}
%
% \section{雑マクロ}
%
% 以下が実際のマクロ集です。
%
%    \begin{macrocode}
%<*okumacro>
%    \end{macrocode}
%
% \begin{macro}{\ruby}
%
% |\ruby{奥}{おく}\ruby{村}{むら}| で\ruby{奥}{おく}\ruby{村}{むら}と
% なります。|\ruby{奥村}{おくむら}| でもかまいませんが,奥と村の間で
% 行分割ができなくなります。
%
% 和文ベースラインが $88:12$ の位置にあることを仮定して支柱を入れました。
%
% バグ: |1\ruby{奥}{おく}1| のような場合に |\xkanjiskip| が入りません。
%
% [2003-02-09] ゴシック体のルビがゴシック体になるバグ?を修正しました。
%
% [2003-05-29] 齋藤修三郎さんのOTFパッケージのexpertオプションに対応しました。
%
% [2003-09-01] オプション引数で幅が指定できるようにしました [2003-10-02バグフィックス]。
%
% [2010-03-14] ルビの直後で改行できなかったので,
% 親字が和文と仮定して直後に |\kanjiskip| を入れました。
%
% [2016-08-16] 縦組に対応。
%
% [2016-08-16] 「前に |\kanjiskip| が入らない」
% 「前後の欧文文字との間に |\xkanjiskip| が入らない」
% 「後ろの禁則処理が効かない」問題を解決するために和文ゴースト処理を追加。
%
%    \begin{macrocode}
\providecommand{\rubyfamily}{}
\def\kanjistrut{\iftdir
  \vrule \@height0.5zw \@depth0.5zw \@width\z@
\else
  \vrule \@height0.88zw \@depth0.12zw \@width\z@
\fi}
\chardef\okumacro@zsp=\jis"2121\relax
\newcommand{\ruby}[3][0zw]{%
  \okumacro@zsp
  \kern-1zw\relax
  \dimen3=\f@size\p@
  \setbox1=\hbox{#2}%
  \setbox3=\hbox{\rubyfamily\fontsize{0.5\dimen3}{0pt}\selectfont #3}%
  \dimen1=#1\relax
  \ifdim\dimen1=\z@
    \ifdim\wd1>\wd3 \dimen1=\wd1 \else \dimen1=\wd3 \fi
  \else
    \ifdim\dimen1<\wd1
      \dimen1=\wd1
    \fi
  \fi
  \hbox{%
    \kanjiskip=0pt plus 2fil
    \xkanjiskip=0pt plus 2fil
    \vbox{%
      \hbox to \dimen1{\rubyfamily
        \fontsize{0.5\dimen3}{0pt}\selectfont \kanjistrut
        \ifdim\dimen1<\wd3
          \hss\unhbox3\hss
        \else
          \hfil\unhbox3\hfil
        \fi
      }%
      \nointerlineskip
      \hbox to \dimen1{\kanjistrut\hfil\unhbox1\hfil}}}%
      \kern-1zw\relax\okumacro@zsp}
%    \end{macrocode}
% \end{macro}
%
% \begin{macro}{\Ruby}
%
% ルビが両側にルビの半文字分ずつはみ出してもいい場合に使います。
% ただ,これを連続して使うと,ルビが重なり合ってしまいます。
%
% これはあまり存在意義がないのでとりあえず消しておきます。
%
%    \begin{macrocode}
% \newcommand{\Ruby}[2]{%
%   \leavevmode
%   \setbox0=\hbox{#1}%
%   \mytempdima=\f@size\p@
%   \setbox1=\hbox{\fontsize{0.5\mytempdima}{0pt}\selectfont #2}%
%   \dimen0=\wd1 \advance \dimen0 by -0.5\mytempdima
%   \ifdim\wd0>\dimen0 \dimen0=\wd0 \fi
%   \hbox{%
%     \kanjiskip=0pt plus 2fil
%     \xkanjiskip=0pt plus 2fil
%     \vbox{%
%       \hbox to \dimen0{\normalfont\rubyfamily
%         \fontsize{0.5\mytempdima}{0pt}\selectfont \kanjistrut\hss#2\hss}%
%       \nointerlineskip
%       \hbox to \dimen0{\kanjistrut\hfil#1\hfil}}}}
%    \end{macrocode}
% \end{macro}
%
% \begin{macro}{\kenten}
%
% \kenten{圏点}を振るコマンドです。
%
% [2016-07-30] ルビと圏点の高さを合わせるつもりでしたが,間違って
% 上シフト量を0.63zwとしていましたので,正しい値0.75zwに直しました。
%
% [2017-03-11] 「前後の欧文文字との間に |\xkanjiskip| が入らない」
% 「後ろの禁則処理が効かない」問題を解決するために和文ゴースト処理を追加。
%
% [2017-10-04] もともと |\kenten| では
%\begin{verbatim}
%    \setbox1=\hbox to \z@{・\hss}%
%\end{verbatim}
% としていましたが,2017年9月の\pTeX (p3.7.2)に対応するため
%\begin{verbatim}
%    \setbox1=\hbox to\z@{\hbox to 1zw{\hss ・\hss}\hss}%
%\end{verbatim}
% に変更しました。これで,p3.7.1以前・p3.7.2以降どちらもOKです。
%
%    \begin{macrocode}
\def\kenten#1{%
  \okumacro@zsp
  \kern-1zw\relax
  \setbox1=\hbox to\z@{\hbox to 1zw{\hss ・\hss}\hss}%
  \ht1=.63zw
  \@kenten#1\end\relax
  \kern-1zw\relax\okumacro@zsp}
\def\@kenten#1#2{%
  \ifx#1\end \let\next=\relax \else
    \raise.75zw\copy1\nobreak #1\ifx#2\end\else\hskip\kanjiskip\relax\fi
    \let\next=\@kenten
  \fi\next#2}
%    \end{macrocode}
% \end{macro}
%
% \begin{macro}{\kintou}
%
% 簡単な均等割りコマンドです。
%
%    \begin{macrocode}
\newcommand{\kintou}[2]{%
  \leavevmode
  \hbox to #1{%
    \kanjiskip=0pt plus 1fil minus 1fil
    \xkanjiskip=\kanjiskip
    #2}}
%    \end{macrocode}
% \end{macro}
%
% \begin{macro}{\曜}
%
% 今日の曜日を出力します。
%
%    \begin{macrocode}
\newcommand{\曜}{{\@tempcnta=\year \@tempcntb=\month
  \ifnum \@tempcntb<3
    \advance \@tempcnta by -1
    \advance \@tempcntb by 12
  \fi
  \multiply \@tempcntb by 13
  \advance \@tempcntb by 8
  \divide \@tempcntb by 5
  \advance \@tempcntb by \@tempcnta
  \divide \@tempcnta by 4
  \advance \@tempcntb by \@tempcnta
  \divide \@tempcnta by 25
  \advance \@tempcntb by -\@tempcnta
  \divide \@tempcnta by 4
  \advance \@tempcntb by \@tempcnta
  \advance \@tempcntb by \day
  \@tempcnta=\@tempcntb
  \divide \@tempcntb by 7
  \multiply \@tempcntb by 7
  \advance \@tempcnta by -\@tempcntb
  \ifcase \@tempcnta 日\or 月\or 火\or 水\or 木\or 金\or 土\fi}}
%    \end{macrocode}
% \end{macro}
%
% \begin{macro}{\挨拶}
%
% 時候の挨拶を出力します。
%
%    \begin{macrocode}
\def\挨拶{\noindent 拝啓\hspace{1zw}\ifcase\month\or
  厳寒\or 春寒\or 早春\or 陽春\or 新緑\or 向暑\or
  猛暑\or 残暑\or 初秋\or 仲秋\or 晩秋\or 初冬\fi
  の候,ますますご清栄のこととお喜び申し上げます。}
%    \end{macrocode}
% \end{macro}
%
% \begin{macro}{\mytt}
%
% |\texttt| では \verb/|/ |!| |@| |*| |-| |+| |=| |[| |]| |:| |;|
% |"| |`| |'| |<| |>| |?| |,| |.| |/| はそのまま出せますが,
% |!| |?| |.| |:| の後は余分な空きが入ります。
% また,|#| |$| |&| |_| |{| |}| は \texttt{\bslash} で逃げれば使えますが,
% そうしても |_| |{| |}| は正しい \texttt{cmtt} フォントになりません。
% さらに,|~| |^| は \texttt{\bslash} を付けても後続の文字のアクセント
% のようになってしまいます(|\~{}|,|\^{}| のようにすれば何とか
% なりますが)。ここで定義する |\mytt| マクロを使えば,
% |{| |}| は \texttt{\bslash} で逃げれば正しい \texttt{cmtt} フォント
% になり,
% |#| |&| |_| |~| |^| は \texttt{\bslash} で逃げなくてもそのまま
% で出力できます。
% |$| もそのまま出力できるようにしていたのですが,
% 数式モードは使えた方が便利なので,外しました。
% また,|\\| で円印が出ます。
% |/|,|\\|,|.|,|@|,|-| の後ろで改行できるようにしてあります。
% |{\mytt ...}| のようにして使います(|\mytt{...}| ではありません)。
% スペースは若干伸縮するようにしました。
% |\mbox{\mytt ...}| のようにして使えば,
% 和文・欧文間のグルーの自動挿入はしません。
%
%    \begin{macrocode}
\def\myallowbreak{\@ifnextchar\egroup{}{\penalty\exhyphenpenalty}{}}
\begingroup
  \catcode`\-=\active
  \catcode`\.=\active
  \catcode`\/=\active
  \gdef\mytth@@k{%
    \catcode`\-=\active \def-{\char`\-\myallowbreak}%
    \catcode`\.=\active \def.{\char`\.\myallowbreak}%
    \catcode`\/=\active \def/{\char`\/\myallowbreak}%
  }
\endgroup
\def\mytt{\ttfamily
  \catcode`\#=12
  \catcode`\&=12
  \catcode`\^=12
  \catcode`\~=12
  \catcode`\_=12
  \chardef\{=`\{
  \chardef\}=`\}
  \chardef\_=`\_
% \def\\{Y\llap=\penalty\exhyphenpenalty{}}% ¥記号
  \def\\{\char"5C\penalty\exhyphenpenalty{}}% バックスラッシュ
% \spaceskip=.5em  plus.25em minus.15em
  \spaceskip=.32em plus.16em minus.09em
  \frenchspacing
  \noautoxspacing
  \mytth@@k}
%    \end{macrocode}
% \end{macro}
%
% これ以外の記号は次のようにして出力できます。
%
% \begin{quote}
%   |\texttt{\char"20}| \qquad \texttt{\char"20} 見える空白 \\
%   |\texttt{\char"5C}| \qquad \texttt{\char"5C} バックスペース \\
%   |\texttt{\char"0B}| \qquad \texttt{\char"0B} 上向き矢印 \\
%   |\texttt{\char"0C}| \qquad \texttt{\char"0C} 下向き矢印 \\
%   |\texttt{\char"0D}| \qquad \texttt{\char"0D} 左右対称なシングルクォート
% \end{quote}
%
% \begin{macro}{\yen}
% \begin{macro}{\BS}
%
% |\yen| は円印,|\BS| はバックスラッシュです。
%
%    \begin{macrocode}
\DeclareRobustCommand{\yen}{{\ooalign{Y\crcr\hss=\hss}}}
\providecommand{\BS}{\texttt{\char"5C}}
%    \end{macrocode}
% \end{macro}
% \end{macro}
%
% \begin{macro}{\bslash}
%
% バックスラッシュは |doc| パッケージなどでの定義に従っておきます。
%
%    \begin{macrocode}
{\catcode`\|=\z@ \catcode`\\=12|gdef|bslash{\}}
%    \end{macrocode}
% \end{macro}
%
% \begin{macro}{\asciibar}
%
% 真ん中が少し途切れた縦棒 \asciibar\ です。
% \verb/|/ に対応するASCII文字です。
%
%    \begin{macrocode}
\newcommand{\asciibar}{%
  \makebox[0.525em]{\rule[-0.2ex]{0.15ex}{0.7ex}\kern-0.15ex
    \rule[0.8ex]{0.15ex}{0.7ex}}}
%    \end{macrocode}
% \end{macro}
%
% \begin{macro}{\−−}
% \begin{macro}{\――}
%
% 倍角のダーシです。昔は |\rule[0.33zh]{2zw}{.03zh}| としていましたが,
% ちょっと低すぎるようです。和文フォントのベースライン
% は $88:12$ のところにあるものが多いので,0.38\,zwのところに中心線が
% 来るほうがいいようです。太さもちょっとだけ太くしました。
%
%    \begin{macrocode}
% \def\−−{\rule[0.36zw]{2zw}{0.04zw}}
%    \end{macrocode}
%
% さらによく考えたら |\scalebox{2}[1]{―}| のほうがよさそうなので,
% |graphics| または |graphicx| パッケージを読み込んだ場合はこちらを使うようにしました。
%
%    \begin{macrocode}
% \def\−−{\@ifundefined{scalebox}{\rule[0.36zw]{2zw}{0.04zw}}{\scalebox{2}[1]{―}}}
%    \end{macrocode}
%
% さらにいろいろなご意見をいただき,次のように直しました。
%
%    \begin{macrocode}
\def\−−{―\kern-.5zw―\kern-.5zw―}
%    \end{macrocode}
%
% 上のものは全角マイナス二つでしたが,全角ダーシ二つでもいいようにしました。
%
%    \begin{macrocode}
\def\――{―\kern-.5zw―\kern-.5zw―}
%    \end{macrocode}
%
% [2016-12-04] \pLaTeX で通るコードが \upLaTeX で通らなくなることを防ぐために,u\pLaTeX の場合にはU+FF0D,U+2014で定義したものも別途用意しました。
%
%    \begin{macrocode}
\ifnum\jis"2121="3000 % upTeX check
  \begingroup
    \kansujichar1="FF0D
    \kansujichar2="2014
    \expandafter\expandafter\expandafter\gdef
    \expandafter\csname\expandafter\kansuji\expandafter1%
    \expandafter\endcsname\kansuji1{―\kern-.5zw―\kern-.5zw―}
    \expandafter\expandafter\expandafter\gdef
    \expandafter\csname\expandafter\kansuji\expandafter2%
    \expandafter\endcsname\kansuji2{―\kern-.5zw―\kern-.5zw―}
  \endgroup
\fi
%    \end{macrocode}
%
% \end{macro}
% \end{macro}
%
%
% \section{キートップ,○数字}
%
% \begin{macro}{\removept}
%
% |#1| に与えたdimenレジスタの pt を消した数値を |#2| のマクロに
% 定義します。
%
%    \begin{macrocode}
\newcommand{\removept}[2]{%
  \edef#2{\expandafter\@removePT\the#1}}
{\catcode`p=12 \catcode`t=12 \gdef\@removePT#1pt{#1}}
% しかし latex.ltx で次のように定義されていました。
% これをそのまま使ったらよかった。
% \begingroup
%   \catcode`P=12
%   \catcode`T=12
%   \lowercase{
%     \def\x{\def\rem@pt##1.##2PT{##1\ifnum##2>\z@.##2\fi}}}
%   \expandafter\endgroup\x
% \def\strip@pt{\expandafter\rem@pt\the}
%    \end{macrocode}
% \end{macro}
%
% \begin{macro}{\keytop}
%
% これを使ってキートップ記号を作ります。
%
% [2016-07-30] |\keytop| はp\LaTeX のascmacパッケージ(tascmac.sty)でも
% 定義されていますが,ascmacの定義では周囲と同じファミリのままになっていて,
% okumacroでは中の文字をサンセリフ体に変更しています。
% 両方のパッケージを読み込む場合は,後に読み込んだほうの定義が勝ちます。
% okumacroが勝つ場合には常にサンセリフ体に変更するキートップ記号を出力しますし,
% ascmacが勝つ場合は周囲と同じファミリになります。
%
% また,新たに |\okukeytop| という命令を追加します。これはascmacと干渉せず,
% 常にサンセリフ体に変更するキートップ記号を出力できます。
%
%    \begin{macrocode}
\newcommand{\okukeytop}[1]{{\leavevmode\kern1pt
  \setbox1=\hbox{\normalfont\sffamily [}%
  \dimen1=\ht1
  \removept{\dimen1}{\tempC}%
  \setbox0=\hbox{\normalfont\sffamily #1}%
  \dp0=0pt
  \dimen0=\wd0
  \ifdim\dimen0<\ht1 \dimen0=\ht1 \fi
  \advance \dimen0 by 5pt
  \removept{\dimen0}{\tempA}%
  \divide \dimen0 by 2
  \removept{\dimen0}{\tempB}%
  \unitlength=1pt
  \advance \dimen1 by 2pt
  \hbox{\vrule height\dimen1 depth2pt width0pt}%
  \begin{picture}(\tempA,\tempC)
    \put(0,0){\line(0,1){\tempC}}
    \put(\tempA,0){\line(0,1){\tempC}}
    \put(\tempB,0){\oval(\tempA,4)[b]}
    \put(\tempB,\tempC){\oval(\tempA,4)[t]}
    \put(\tempB,0){\makebox(0,0)[b]{\box0}}
  \end{picture}\kern1pt}}
\expandafter\ifx\csname ver@tascmac.sty\endcsname\relax
  \newcommand{\keytop}{\okukeytop}
\else
  \renewcommand{\keytop}{\okukeytop}
\fi
%    \end{macrocode}
% \end{macro}
%
% \begin{macro}{\return}
%
% リターンキーの記号です。
%
% [2016-07-30] |\return| はp\LaTeX のascmacパッケージ(tascmac.sty)でも
% 定義されていますが,okumacroがリターンのキートップ記号を出力するのに対し,
% ascmacではリターンの矢印文字{\font\ASCGRP=ascgrp \ASCGRP\char"20}だけ
% を出す命令になっています。
% 両方のパッケージを読み込む場合は,後に読み込んだほうの定義が勝ちます。
% okumacroが勝つ場合には\okureturn を出力しますし,
% ascmacが勝つ場合は{\font\ASCGRP=ascgrp \ASCGRP\char"20}を出力します。
%
% また,新たに |\okureturn| という命令を追加します。これはascmacと干渉せず,
% 常にリターンのキートップ記号\okureturn を出力できます。
%
%    \begin{macrocode}
\def\RETMARK{{\unitlength=1pt
  \setbox0=\hbox{\normalfont\ttfamily [}%
  \dimen0=\ht0
  \removept{\dimen0}{\tempA}%
  \advance \dimen0 by -1pt
  \removept{\dimen0}{\tempB}%
  \begin{picture}(\tempA,\tempA)
    \put(\tempA,1){\line(0,1){\tempB}}
    \put(\tempA,1){\vector(-1,0){\tempA}}
  \end{picture}}}
\newcommand{\okureturn}{\okukeytop{\RETMARK}}
\expandafter\ifx\csname ver@tascmac.sty\endcsname\relax
  \newcommand{\return}{\okureturn}
\else
  \renewcommand{\return}{\okureturn}
\fi
%    \end{macrocode}
% \end{macro}
%
% \begin{macro}{\upkey}
% \begin{macro}{\downkey}
% \begin{macro}{\rightkey}
% \begin{macro}{\leftkey}
%
%    上下右左の矢印キーです。
%
%    \begin{macrocode}
\newcommand{\upkey}{\raisebox{-2pt}{%
  \setlength{\unitlength}{1pt}%
  \thinlines
  \begin{picture}(14,9)
    \put(7,2){\vector(0,1){8}}
    \put(1,2){\line(0,1){7}}
    \put(13,2){\line(0,1){7}}
    \put(7,9){\oval(12,4)[t]}
    \put(7,2){\oval(12,4)[b]}
  \end{picture}%
}}
\newcommand{\downkey}{\raisebox{-2pt}{%
  \setlength{\unitlength}{1pt}%
  \thinlines
  \begin{picture}(14,9)
    \put(7,10){\vector(0,-1){8}}
    \put(1,2){\line(0,1){7}}
    \put(13,2){\line(0,1){7}}
    \put(7,9){\oval(12,4)[t]}
    \put(7,2){\oval(12,4)[b]}
  \end{picture}%
}}
\newcommand{\rightkey}{\raisebox{-2pt}{%
  \setlength{\unitlength}{1pt}%
  \thinlines
  \begin{picture}(14,9)
    \put(3,6){\vector(1,0){8}}
    \put(1,2){\line(0,1){7}}
    \put(13,2){\line(0,1){7}}
    \put(7,9){\oval(12,4)[t]}
    \put(7,2){\oval(12,4)[b]}
  \end{picture}%
}}
\newcommand{\leftkey}{\raisebox{-2pt}{%
  \setlength{\unitlength}{1pt}%
  \thinlines
  \begin{picture}(14,9)
    \put(11,6){\vector(-1,0){8}}
    \put(1,2){\line(0,1){7}}
    \put(13,2){\line(0,1){7}}
    \put(7,9){\oval(12,4)[t]}
    \put(7,2){\oval(12,4)[b]}
  \end{picture}%
}}
%    \end{macrocode}
% \end{macro}
% \end{macro}
% \end{macro}
% \end{macro}
% 
% \begin{macro}{\MARU}
%
% 丸で囲んだ文字です。
%
% [2003-01-24] |graphicx| のとき少し見映えが良くなるようにしました。
%
%    \begin{macrocode}
% \newcommand{\MARU}[1]{{%
%   \ooalign{\hfil#1\/\hfil\crcr\raise.167ex\hbox{\mathhexbox20D}}}}
\newcommand{\MARU}[1]{{%
    \ooalign{\hfil
      \@ifundefined{scalebox}{%
        #1\/\hfil\crcr\raise.167ex}{\raise.1zw\hbox{\scalebox{0.8}{#1\/}}\hfil\crcr\raise.1zw}%
      \hbox{\fontencoding{OMS}\fontfamily{cmsy}\fontseries{m}\fontshape{n}\selectfont
        \char"0D}}}}
%    \end{macrocode}
% \end{macro}
%
% その他のロゴです。
%
%    \begin{macrocode}
% \newcommand{\MuTeX}{M\kern-.15em\raise.2ex\hbox{u}\kern-.32em\TeX}
\newcommand{\PiC}{P\kern-.12em\lower.5ex\hbox{I}\kern-.075emC}
\newcommand{\PiCTeX}{\PiC\kern-.11em\TeX}
\newcommand{\JTeX}{\leavevmode\lower.5ex\hbox{J}\kern-.17em\TeX}
\newcommand{\JLaTeX}{\leavevmode\lower.5ex\hbox{\rm J}\kern-.17em\LaTeX}
\DeclareRobustCommand{\JBibTeX}{$\m@th\mbox{%
  \lower.5ex\hbox{J}\kern-.05em\BibTeX}$}
\newcommand{\pTeXsT}{p\kern-.21em\TeX\kern-.10em s\kern-.21em T}
% インプレスの新 TeX の仮称
\newcommand{\iTeX}{\accent'27\i\TeX}
% 桜井さんの新 TeX のロゴ(仮)
\newcommand{\MlTeX}{M\kern-.1667em\lower.5ex\hbox{L}\kern-.25em\TeX}
%    \end{macrocode}
%
% METAFONTのロゴを出す方法はいろいろあります。まず古い方法で,
% |manfnt| を使うものです。
%
%\begin{verbatim}
% \font\manual=manfnt
% \font\manLarge=manfnt scaled \magstep 2
% \font\manHuge=manfnt scaled \magstep 5
% \newcommand{\MF}{%
%   \ifx\@currsize\small
%     {\manual hijk}\-{\manual lmnj}% 9pt
%   \else
%     \ifx\@currsize\footnotesize
%       \ifdim \fontdimen\@ne\font>\z@
%         {\manual \char92\char93\char94\char95efg\char94}% 8pt slanted
%       \else
%         {\manual opqr}\-{\manual stuq}% 8pt
%       \fi
%     \else
%       \ifx\@currsize\Large
%         {\manLarge META}\-{\manLarge FONT}% mag2 (14pt)
%       \else
%         \ifx\@currsize\Huge
%           {\manHuge META}\-{\manHuge FONT}% mag5 (25pt)
%         \else % \normalsize
%           \ifdim \fontdimen\@ne\font>\z@
%             {\manual 89:;<=>:}% 10pt slanted
%           \else
%             {\manual META}\-{\manual FONT}% 10pt
%           \fi
%         \fi
%       \fi
%     \fi
%   \fi}
%\end{verbatim}
%
% 次は |logo| を使う方法です。
%
%\begin{verbatim}
%\font\logoTen=logo10
%\font\logoEight=logo8
%\font\logoLarge=logo10 scaled \magstep 2
%\font\logoHuge=logo10 scaled \magstep 5
%\newcommand{\MF}{%
%  \ifx\@currsize\Large
%    {\logoLarge META}\-{\logoLarge FONT}%
%  \else
%    \ifx\@currsize\Huge
%      {\logoHuge META}\-{\logoHuge FONT}%
%    \else
%       \ifx\@currsize\footnotesize
%         {\logoEight META}\-{\logoEight FONT}%
%       \else
%         {\logoTen META}\-{\logoTen FONT}%
%       \fi
%    \fi
%  \fi}
%\end{verbatim}
%
% \LaTeXe の方法としては,
% 次のような |ulogo.fd| を作っておきます。
%
%\begin{verbatim}
%\DeclareFontFamily{U}{logo}{}
%\DeclareFontShape{U}{logo}{m}{n}{%
%      <5><6><7><8>logo8%
%      <9>logo9%
%      <10><10.95><12><14.4><17.28><20.74><24.88>logo10}{}
%\DeclareFontShape{U}{logo}{m}{sl}{%
%      <5><6><7><8><9>logosl9%
%      <10><10.95><12><14.4><17.28><20.74><24.88>logosl10}{}
%\DeclareFontShape{U}{logo}{bx}{n}{%
%      <5><6><7><8><9>%
%      <10><10.95><12><14.4><17.28><20.74><24.88>logobf10}{}
%\DeclareFontShape{U}{logo}{m}{it}
%  {<->ssub*logo/m/sl}{}
%\end{verbatim}
%
% これで次のようにします。
%
%\begin{verbatim}
% \DeclareRobustCommand{\MF}{{\fontfamily{logo}\selectfont META}\-%
%   {\fontfamily{logo}\selectfont FONT}}
% \DeclareRobustCommand{\MP}{{\fontfamily{logo}\selectfont META}\-%
%   {\fontfamily{logo}\selectfont POST}}
%\end{verbatim}
%
% しかしよく調べてみたら同様なものが |mflogo.sty|,|ulogo.fd|
% として配布されていましたので,そちらを使うことにします。
%
% \section{いろいろな環境}
%
% \begin{environment}{dangerous}
%
% 参考事項をメモ風に付ける環境です。鉛筆記号を使うようにしました。
%
%    \begin{macrocode}
%\newcommand{\dangerouslabel}[1]{{%
%  \setlength{\fboxsep}{1pt}%
%  \fbox{\footnotesize\bfseries #1}}}
\newenvironment{dangerous}{%
  \vspace{3pt}%
  \begin{list}{\ding{"2E}}{%
%   \let\makelabel=\dangerouslabel
    \setlength{\leftmargin}{3zw}%
    \setlength{\itemindent}{0pt}%
    \setlength{\labelwidth}{2zw}%
    \setlength{\labelsep}{6pt}%
    \setlength{\rightmargin}{0pt}%
    \setlength{\itemsep}{0.5\baselineskip}%
    \setlength{\parsep}{0pt}%
    \setlength{\topsep}{0.5\baselineskip}%
    \setlength{\partopsep}{0pt}%
    \setlength{\listparindent}{0zw}%
  }\small}{\end{list}}
%    \end{macrocode}
% \end{environment}
%
% \begin{environment}{namelist}
%    |\begin{namelist}{xxxx} \item[...] ... \end{namelist}| で
%    見出しを |xxxx| の幅にしたリストを作ります。
%    \begin{macrocode}
% \newcommand{\namelistlabel}[1]{{#1}\ \dotfill\ }
\newcommand{\namelistlabel}[1]{\hspace{1zw}#1\hfil}
\newenvironment{namelist}[1]{%
  \begin{list}{}{%
    \let\makelabel=\namelistlabel
    \settowidth{\labelwidth}{#1}%
    \addtolength{\labelwidth}{2zw}%
    \setlength{\leftmargin}{\labelwidth}%
    \setlength{\labelsep}{0pt}%
  }}{\end{list}}
%    \end{macrocode}
% \end{environment}
%
% \begin{environment}{mybibliography}
%
% |thebibliography| とほぼ同じ参考文献リスト環境ですが,
% 章(節)見出しを出力せず,カウンタをリセットしません。
% |thebibliography| と同じ |enumiv| というカウンタを使っていますので,
% 必要に応じて
% \begin{quote}
%   |\setcounter{enumiv}{0}|
% \end{quote}
% として初期化してください。
%
%    \begin{macrocode}
\newenvironment{mybibliography}[1]{%
   \list{\@biblabel{\@arabic\c@enumiv}}%
        {\settowidth\labelwidth{\@biblabel{#1}}%
         \leftmargin\labelwidth
         \advance\leftmargin\labelsep
         \@openbib@code
         \@nmbrlisttrue\def\@listctr{enumiv}%
         \let\p@enumiv\@empty
         \renewcommand\theenumiv{\@arabic\c@enumiv}}%
   \sloppy
   \clubpenalty4000
   \@clubpenalty\clubpenalty
   \widowpenalty4000%
   \sfcode`\.\@m}
  {\def\@noitemerr
    {\@latex@warning{Empty `mybibliography' environment}}%
   \endlist}
%    \end{macrocode}
% \end{environment}
%
% \begin{environment}{FRAME}
%
% 枠です。
%
%    \begin{macrocode}
\newenvironment{FRAME}{\begin{trivlist}\item[]
  \hrule width\linewidth
  \hbox to \linewidth\bgroup
    \advance\linewidth by -30pt
    \hsize=\linewidth
    \vrule\hfill
    \vbox\bgroup
      \vskip15pt
      \def\thempfootnote{\arabic{mpfootnote}}% 脚注を算用数字に
      \begin{minipage}{\linewidth}}{%
      \end{minipage}\vskip15pt
    \egroup\hfill\vrule
  \egroup\hrule width\linewidth
  \end{trivlist}}
\newcommand{\SHUTTEN}[1]{%
  \par\nobreak\removelastskip\rightline{\small#1}\medbreak}
%    \end{macrocode}
% \end{environment}
%
% \begin{environment}{screen}
%
% スクリーン風の環境です。元は |ascmac.sty| にあったのだと
% 思いますが,かなり改変したかもしれません。
%
% |\begin{screen}[0.75]| のようにオプションの数値を与えると
% 横幅を0.75倍するようにしてみました。
% 倍率が 1 以外の場合は \texttt{graphicx} パッケージが必要です。
% 画面出力例がコラム幅を超える場合の対処法として有効かもしれません。
% ただ,screen環境の最初に |[| が来る場合は誤動作しますので,|\relax|
% か |{}| を入れて誤魔化してください。
%
% [2016-07-30] 元になっているp\LaTeX のascmacパッケージ(tascmac.sty)の
% screen環境は,オプション引数の数値で角の丸みの大きさを整数値(0--8)で指定し
% ますが,okumacroパッケージでは横の倍率を指定するという違いがあります。
% 両方のパッケージを読み込む場合は,後に読み込んだほうの定義が勝ちます。
% screen環境は,okumacroが勝つ場合には横の倍率を指定できる
% スクリーン風の環境になりますし,ascmacが勝つ場合は角の丸みを指定できる
% スクリーン風の環境になります。
%
% [2016-10-08] 画面出力例がコラム幅を超える場合の対処法として横倍率指定の
% オプションを追加しましたが,同じ枠内に改行可能な文章も含まれる場合に
% 枠の右端から離れた不自然な位置(横倍率を縮めない場合に端にあたっていた
% 位置)で改行していました。これを修正しました。
%
% また,新たにokuscreen環境を追加します。これはascmacと干渉せず,常に横の
% 倍率を指定できるスクリーン風の環境になります。
%
% [2018-06-11] 2017/10/04版までは,枠の中身の深さを考慮していなかったので,
% 表(tabular)や深さのある数式を囲むと枠からはみ出していました。この問題を
% 修正しましたが,なるべく既存文書のレイアウトが変わらないように,通常の
% テキストの深さ分だけ引いておきました(|!| と印をつけた部分)。
%
%    \begin{macrocode}
\expandafter\ifx\csname ver@tascmac.sty\endcsname\relax
  \newdimen\@scw
  \newbox\scb@x
\fi
\newenvironment{okuscreen}[1][1]%
 {\def\screensc@le{#1}\@scw=\linewidth \advance \@scw by -20pt
  \dimen1=#1\p@\relax
  \@tempcnta=\dimen1\relax
  \@tempcntb=65536\relax
  \divide\@scw by \@tempcnta
  \multiply\@scw by \@tempcntb
  \setbox\scb@x=\hbox\bgroup\begin{minipage}[b]{\@scw}}%
     % または \setbox\scb@x=\vbox\bgroup\advance \linewidth by -20pt \relax
 {\end{minipage}\egroup
  \dimen0=\linewidth \removept{\dimen0}{\hsPT}%
  \divide \dimen0 by 2 \removept{\dimen0}{\hhsPT}%
  \dimen0=\dp\scb@x \advance \dimen0 by 5pt
  \advance \dimen0 by -.5\cdp %% !
  \removept{\dimen0}{\hdxPT}%
  \advance \dimen0 by \ht\scb@x \advance \dimen0 by 5pt
  \advance \dimen0 by -.5\cdp %% !
  \removept{\dimen0}{\htxPT}%
  \begin{flushleft}
    \vspace{6.5pt}%
    \begin{picture}(\hsPT,\htxPT)
      \put(\hhsPT,\htxPT){\oval(\hsPT,10)[t]}
      \put(0,0){\line(0,1){\htxPT}}
      \put(\hsPT,0){\line(0,1){\htxPT}}
      \put(\hhsPT,0){\oval(\hsPT,10)[b]}
      \put(10,\hdxPT){\ifdim\screensc@le pt=1pt \box\scb@x
                 \else\scalebox{\screensc@le}[1]{\box\scb@x}\fi}
    \end{picture}%
  \end{flushleft}}
\let\screen\okuscreen \let\endscreen\endokuscreen
%    \end{macrocode}
% \end{environment}
%
% \begin{environment}{EXAMPLE}
%
% 出力例です。ずいぶん前に作ったので,再検討の要があります。
%
%    \begin{macrocode}
\newdimen\EXAMPLEWIDTH
\newenvironment{EXAMPLE}{\begin{flushleft}\EXAMPLEWIDTH=\linewidth
  }{\end{flushleft}}
\newenvironment{IN}{%
  \makebox[35pt][l]{\small\fbox{入力}}%
  \advance \linewidth by -35pt
  \begin{minipage}[t]{\linewidth}}{%
  \end{minipage}\\[6pt]}
\newenvironment{OUT}{%
  \makebox[35pt][l]{\small\fbox{出力}}%
  \advance \linewidth by -35pt
  \begin{minipage}[t]{\linewidth}}{%
  \end{minipage}}
\newcommand{\INEX}{\medbreak\noindent
  \framebox[\columnwidth][l]{\small\hskip2zw 入力}\nobreak
  \begin{quote}\small}
\newcommand{\OUTEX}{\end{quote}
  \framebox[\columnwidth][l]{\small\hskip2zw 出力}\nobreak
  \begin{quotation}\small}
\newcommand{\ENDEX}{\end{quotation}\bigbreak}
\newcommand{\ENDEXC}[1]{\end{quotation}\nobreak
  \begin{flushright}\footnotesize#1\end{flushright}\bigbreak}
%    \end{macrocode}
% \end{environment}
%
% \section{雑}

% \begin{macro}{\whichpage}
%
% |\whichpage{参照名}| で,そのページが同じページにあれば何も出力せず,
% そうでなければ(次ページ)(前ページ)(123ページ)のような形式で
% 場所を出力します。
%
% [2025-05-10] \LaTeXe~2022-06-01以降で
% |! Missing number, treated as zero.|エラーが
% 出るようになっていたので,修正します。
% まず「展開可能版の |\pageref|」を別に用意して,
% |\whichpage| の実装中でそれを使うようにしました。
%
%    \begin{macrocode}
\def\okumacro@pageref#1{%
  \expandafter\expandafter\expandafter
      \okumacro@pageref@a\csname r@#1\endcsname{0}\@nil}
\def\okumacro@pageref@a#1#2#3\@nil{#2}
\def\whichpage#1{%
  \expandafter\ifx\csname r@#1\endcsname\relax
  \else
    \@tempcnta\okumacro@pageref{#1}\relax
    \advance\@tempcnta-\c@page
    \ifnum\@tempcnta=0
    \else
      \ifnum\@tempcnta=1 (次ページ)
      \else
        \ifnum\@tempcnta=-1 (前ページ)
        \else (\okumacro@pageref{#1}ページ)
        \fi
      \fi
    \fi
  \fi}
%    \end{macrocode}
% \end{macro}
%
% \begin{environment}{migiake}
%
%    ごく簡単な右空けコマンドです。使っていません。
%
%    \begin{macrocode}
% \newenvironment{migiake}[1]{%
%   \setlength{\rightskip}{#1}\samepage}{\par}
%    \end{macrocode}
% \end{environment}
%
% \begin{macro}{\rightfig}
%
%    もう少しましな右空けです。
%    |#1| は空け幅, |#2| は行数, |#3| は図や表です。
%
%    \begin{macrocode}
% \def\rightfig#1#2#3{\noindent
%   \dimen0=\baselineskip \multiply \dimen0 by #2
%   \advance\dimen0 by -\baselineskip \advance\dimen0 by 2ex
%   \hangindent=-#1 \hangafter=-#2
%   \raise 2ex \hbox to 0pt{\smash{\vtop to \dimen0{\vfil
%   \vbox{\hfill \hbox to #1{ \ \hfill #3}}\vfil
%   }}\hss}\hskip\parindent\ignorespaces}
%    \end{macrocode}
% \end{macro}
%
% 漢字の幅をttフォント2文字分にします。使っていません。
%
%    \begin{macrocode}
% \newcommand{\kanjitt}[1]{\texttt{\kanjiskip0.083595em\relax#1}}
%    \end{macrocode}
%
% \begin{environment}{shadebox}
%
% 影付きの箱です。|ascmac.sty| から取りました。
%
% [2016-07-30] 元になっているp\LaTeX のascmacパッケージ(tascmac.sty)と
% 衝突しないように,新しい仕様では,ascmacとokumacroの両方を読み込む場合には
% 「後に読み込んだほう」が勝ちます。
%
% また,新たにokushadebox環境を追加します。これはascmacと干渉せず,常に
% okumacro流の定義になります。
%
%    \begin{macrocode}
\expandafter\ifx\csname ver@tascmac.sty\endcsname\relax
  \newdimen\shaderule \shaderule 5\p@
\fi
\def\okushadebox{\dimen0\linewidth \advance\dimen0-20\p@ 
  \advance\dimen0-2\fboxrule \advance\dimen0-\shaderule
  \setbox\@tempboxa=\hbox\bgroup\minipage{\dimen0}}
\def\endokushadebox{\endminipage\egroup\dimen0=10\p@ \advance\dimen0-\fboxrule
  \setbox\@tempboxa=\hbox{\kern\dimen0\unhbox\@tempboxa\kern\dimen0}%
  \setbox0=\vbox{\hrule\@height \fboxrule
  \hbox{\vrule\@width \fboxrule \hskip-\fboxrule
  \vbox{\vskip\fboxsep \box\@tempboxa\vskip\fboxsep}%
  \hskip-\fboxrule\vrule \@width \fboxrule}\hrule \@height \fboxrule}%
  \dimen0=\ht0 \advance\dimen0-\shaderule
  \hbox{\hbox to \shaderule{\copy0\hss}\kern \z@
  \vrule\@width\wd0\@height\z@\@depth\shaderule\hskip-\shaderule
  \vrule\@width\shaderule\@height\dimen0}}
\let\shadebox\okushadebox \let\endshadebox\endokushadebox
%    \end{macrocode}
% \end{environment}
%
% \begin{environment}{sankou}
%
% 参考(dangerous bend 風)の環境です。
%
%    \begin{macrocode}
\def\sankoumark{{\small ★}}
\newenvironment{sankou}{\vspace{3pt}%
  \begin{list}{}{%
    \setlength{\leftmargin}{3zw}%
    \setlength{\itemindent}{0pt}%
    \setlength{\labelwidth}{2zw}%
    \setlength{\labelsep}{6pt}%
    \setlength{\rightmargin}{0pt}%
    \setlength{\itemsep}{0pt}%
    \setlength{\parsep}{0pt}%
    \setlength{\topsep}{0.5\baselineskip}%
    \setlength{\partopsep}{0pt}%
    \setlength{\listparindent}{0zw}%
  }\item[\sankoumark]\small}{\end{list}}
%    \end{macrocode}
% \end{environment}
%
% \begin{environment}{toi}
%
% 問いの環境です。
%
%    \begin{macrocode}
\def\toimark{{\small\bf 【問】}}
\newenvironment{toi}{\vspace{3pt}%
  \begin{list}{}{%
    \setlength{\leftmargin}{3zw}%
    \setlength{\itemindent}{0pt}%
    \setlength{\labelwidth}{2zw}%
    \setlength{\labelsep}{6pt}%
    \setlength{\rightmargin}{0pt}%
    \setlength{\itemsep}{0pt}%
    \setlength{\parsep}{3pt plus 1pt minus 1pt}%
    \setlength{\topsep}{3pt plus 1pt minus 1pt}%
    \setlength{\partopsep}{3pt plus 6pt minus 1pt}%
    \setlength{\listparindent}{0zw}%
  }\item[\toimark]\small}{\end{list}}
%    \end{macrocode}
% \end{environment}
%
% 英語用の行送りに切り替えるコマンドですが,使っていません。
%
%    \begin{macrocode}
% \def\english{%
%   \ifx\@currsize\normalsize   \fontsize{10}{12pt}\selectfont\else
%   \ifx\@currsize\small        \fontsize{9}{11pt}\selectfont\else
%   \ifx\@currsize\footnotesize \fontsize{8}{9.5pt}\selectfont\fi\fi\fi}
%    \end{macrocode}
%
%    \begin{macrocode}
% \newcommand{\shima}{dvi\-out\slash dvi\-prt\index{dviout/dviprt}}
\providecommand{\eps}{\varepsilon}
% \newcommand{\zip}{〒\kern0pt\relax}
% \def\mathpalette#1#2{\mathchoice{#1\displaystyle{#2}}%
%   {#1\textstyle{#2}}{#1\scriptstyle{#2}}{#1\scriptscriptstyle{#2}}}
% ≦, ≧,=の上下に点
\def\LEQQ{\mathrel{\mathpalette\gl@align<}}
\def\GEQQ{\mathrel{\mathpalette\gl@align>}}
\def\gl@align#1#2{\lower.6ex\vbox{\baselineskip\z@skip\lineskip\z@
    \ialign{$\m@th#1\hfil##\hfil$\crcr#2\crcr=\crcr}}}
\def\APPROX{\mathrel{\mathpalette\ap@align{\smash.}}}
\def\ap@align#1#2{\lower.2ex\vbox{\baselineskip\z@skip\lineskip\z@
    \def\finsm@sh{\ht\z@.2ex \dp\z@.2ex \box\z@}%
    \ialign{$\m@th#1\hfil##\hfil$\crcr#2\crcr=\crcr#2\crcr}}}
%%% Fractions like 1/2 by DEK (TUGboat, 6(1):36, TeXbook (Excercise 11.6))
\def\FRAC#1#2{\leavevmode\kern.1em
  \raise.5ex\hbox{\the\scriptfont0 #1}\kern-.1em
  /\kern-.15em\lower.25ex\hbox{\the\scriptfont0 #2}}
\newcommand{\hk}{\hspace{\kanjiskip}}
\newcommand{\hx}{\hspace{\xkanjiskip}}
\newcommand{\3}{\penalty300 }
\newcommand{\5}{\penalty50 }          % \penalty\exhyphenpenalty
%    \end{macrocode}
%
% \begin{macro}{\>}
% 全角コンマの直後に半角文字が来た場合,全角コンマの後に挿入して,
% 半角空白と同じだけの伸縮を持たせるためのグルー挿入コマンドです。
% 使っていません。
%    \begin{macrocode}
% \newcommand{\>}{\relax
%   \ifmmode\mskip\medmuskip\else\hskip0em plus0.167em minus0.111em\fi}
%    \end{macrocode}
% \end{macro}
%
% \begin{macro}{\ANGLE}
% Angle brackets です。
%    \begin{macrocode}
\newcommand{\ANGLE}[1]{%
  $\langle$\kern0pt\nobreak{#1}\kern0pt\nobreak$\rangle$}
%    \end{macrocode}
% \end{macro}
%
% 分数関係です。\texttt{amstex} パッケージでもほぼ同じものが
% 定義されていますが,\texttt{amstex} パッケージの |\cfrac| で
% は |\mathstrut| ではなく |\strut| が使われているため,
% 和文スタイルで行送りが大きい場合には,ちょっと間延びしてしまいます。
%    \begin{macrocode}
% \providecommand{\tfrac}[2]{{\textstyle\frac{#1}{#2}}}
% \providecommand{\dfrac}[2]{{\displaystyle\frac{#1}{#2}}}
% \providecommand{\cfrac}[2]{\dfrac{\mathstrut #1}{#2}}
%    \end{macrocode}
%
% \begin{macro}{\dalembert}
% D'Alembertian(|\Box| より少々大きい四角)です。
%    \begin{macrocode}
% \newcommand{\dalembert}{{%
%   \setlength{\unitlength}{0.1em}%
%   \begin{picture}(8,8)
%     \thinlines
%     \put(1,0){\framebox(6,6){}}
%   \end{picture}}}
%    \end{macrocode}
% \end{macro}
% 
% \begin{macro}{\kanjibox}
% 漢字の大きさの箱です。使っていません。
%    \begin{macrocode}
% \newcommand{\kanjibox}{{\setlength{\unitlength}{0.1091175zw}%
%   \begin{picture}(9.62216,7.77588)\thinlines
%   \put(0,-1.38855){\framebox(9.16443,9.16443){}}\end{picture}}}
%    \end{macrocode}
% \end{macro}
% 
% Poor person's blackboard boldです
% \footnote{Robert Messer, ``Blackboard Bold.''
%   \textit{TUGboat} 9(1):19--20, 1988.}。
% 
% \begin{macro}{\R}
% \begin{macro}{\C}
%    \begin{macrocode}
% \def\R{{\rm I\kern-.17em R}} % (-.25em in the above literature)
% \def\C{{\rm\kern.24em\vrule width.02em height1.4ex depth-.05ex\kern-.26em C}}
%    \end{macrocode}
% \end{macro}
% \end{macro}
% 
% \begin{macro}{\株}
%    \begin{macrocode}
\newcommand{\株}{\hspace{\kanjiskip}\<(株)\<\hspace{\kanjiskip}}
%    \end{macrocode}
% \end{macro}
% 
% \begin{macro}{\CVS}
%
% CVS用。
%
%    \begin{macrocode}
% \def\CVS$#1: #2 ${\expandafter\def\csname CVS#1\endcsname{#2}}
%    \end{macrocode}
% \end{macro}
%
%\begin{verbatim}
% \CVS$Revision: 1.7 $
% \CVS$Date: 2000/06/09 05:55:02 $
% この文書のバージョンは\CVSRevision 版である。
%\end{verbatim}
% のように使います。
%
% [2003-02-16..17] |js*| クラスで数式の前後のスペースが若干変わってしまいます。
% とりあえず |amsmath| の |equation(*)|,|align(*)| 環境の類だけここで直しておきます。
% このあたりはまた見直します。
%
% [2021-07-01] |okumacro| が有ると |xalignat| 上部のアキが大きくなる現象
% (forum:3029)が報告されました。|js*| クラスのための修正だったはずですが,
% |js*| クラスを使った場合でも「修正しない方がバランスが良い」という挙動
% ^^A 少なくとも\TeX\ Live 2014--2021は同じ挙動。
% ^^A https://oku.edu.mie-u.ac.jp/tex/mod/forum/discuss.php?d=3029
% だったため,削除しました。
%
%    \begin{macrocode}
%\@ifundefined{widebaselines}{}{%
%  \let\endmathdisplaysave=\endmathdisplay
%  \def\endmathdisplay{\widebaselines\endmathdisplaysave}}
%  \let\start@alignsave=\start@align
%  \def\start@align{\vspace{4pt}\start@alignsave}
%    \end{macrocode}
%
% 以上で終わりです。
%
%    \begin{macrocode}
%</okumacro>
\endinput
%    \end{macrocode}
%
% ^^A \endinput の後に \CharacterTable を書いておけば,子ファイルには
% ^^A 出力されません。チェックは *.dtx を処理したときにしてくれます。
% ^^A [2016-09-03] →削除しました。
%
% \Finale
